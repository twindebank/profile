\cvsection{Experience}
\begin{cventries}
    \cventry
    {Data Engineer}
    {BBC}
    {London, UK}
    {Nov 2017 – Current}
    {
      \begin{_cvitems}
        \item{Building a microservice-based infrastructure, designed to ingest and relate content from across the BBC into a single place.}
        \item{Growth of knowledge beyond code is facilitated through exposure to the product growing from concept to reality; for example, how internal/external communications are handled, how roadmaps are developed and are dynamic, how to deal with constantly changing requirements, and how to make decisions in unfamiliar environments.}
        \item{Talking at PyData London 2018 about some of the technologies the team are working with proved an invaluable experience, which, among other things, emphasised to me the importance of the communities that surround the open source tools we use daily.}
      \end{_cvitems}
    }
    \cventry
    {Software Engineer}
    {Cue Sense Ltd.}
    {London, UK}
    {Jun. 2017 – Nov. 2017}
    {
      \begin{_cvitems}
        \item {Built a code base to help visually impaired people to interpret non-verbal cues, utilising computer vision and machine learning.}
        \item {Constructed the entire image processing, image classification, and binaural sound feedback pipeline for the software prototype, using Python 3.6 and a multitude of advanced data science techniques. This has built upon my experience of creating unique machine learning pipelines for novel applications.}
        \item {Designed around challenging physical and economical constraints, helping expand my knowledge and practical experience in designing software architecture and core algorithms for market-facing products.}
        \item {Pitched the product to potential investors, heightening my awareness and understanding of the marketing and business aspects to developing a novel, innovative product.}
      \end{_cvitems}
    }
    \cventry
    {Software Developer Intern}
    {Machine Learning Research Group}
    {University of Oxford}
    {Jul. 2016 – Sep. 2016}
    {
      \begin{_cvitems}
        \item {Used a variety of machine learning techniques during my internship within the Oxford-Man Institute of Quantitative Finance, working on a project to detect mosquito presence, species, and gender from audio recordings.}
        \item {Developed a fully-featured python package to act as a test-bed for detection algorithms, aimed for a public open source release. This has improved my skills in coding for a long-term project where the code will be further used and worked on by others.}
        \item {Planned and carried out a series of microphone tests, resulting in a large batch of microphones being purchased and used to capture further biological recordings.}
        \item {Built upon this work within my fourth year university project, recieving a mark of 83 on final grading.}
        \item{Feedback from supervisors placed me in the top 5\% of students who have worked with the Oxford-Man institute, confirming my ability to carry out high value work in this sector.}
      \end{_cvitems}
    }
    \cventry
    {Software Developer}
    {Engineers Without Borders}
    {Oxford, UK}
    {Nov. 2015 – Mar. 2016}
    {
      \begin{_cvitems}
        \item {Built a data collection system for a team to gather and process information on malaria cases in rural Peru with a PHP/MySQL back-end, improving my project management and team-work skills.}
      \end{_cvitems}
    }
  \cventry
    {Research Assistant}
    {Communications, Sensors, Signal and Information Processing Research Group}
    {Newcastle University, UK}
    {Jul. 2015 – Sep. 2015}
    {
      \begin{_cvitems}
        \item {Internship working on various sonar systems to image undersea objects beneath the seabed.}
        \item {Implemented a variety of processing techniques in MATLAB including delay-sum beam-forming, synthetic
aperture focusing and signal filtering. Challenging problems encountered during the project enhanced my problem solving and research skills.}
      \end{_cvitems}
    }
\end{cventries}
